

\input{header_slides.tex}

\begin{document}


\title[Benchmarking road network growth models]{Benchmarking road network growth models}

\author[Raimbault]{J.~Raimbault$^{1,2,3,\ast}$\\\medskip
$^{\ast}$\texttt{j.raimbault@ucl.ac.uk}
}

\institute[UCL]{$^{1}$Center for Advanced Spatial Analysis, University College London\\
$^{2}$UPS CNRS 3611 Complex Systems Institute Paris\\
$^{3}$UMR CNRS 8504 G{\'e}ographie-cit{\'e}s
}


\date[04/11/2021]{ECTQG 2021\\
Special Session: Exploration and validation of spatial simulation models\\
November 4th 2021
}

\frame{\maketitle}



\section{Introduction}

\sframe{Modeling road network growth}{


% Processes underlying the growth of road networks are diverse and complementary, as for example with the combination of self-organisation and top-down planning (Barthelemy et al., 2013). Multiple generative models, more or less parsimonious and data-driven, have been introduced in the literature to reproduce existing networks and provide potential explanations on main processes driving their growth

\cite{cats2021multi}

\cite{szell2021growing}

\cite{cats2020modelling}



}


\sframe{}{

% Whereas each model includes plausible mechanisms and often yields reasonable empirical results, a systematic and quantitative comparison of such models remains to be explored. We propose in this contribution such a benchmark of road network growth models.

}



\sframe{Road network generation multi-model}{


% Whereas each model includes plausible mechanisms and often yields reasonable empirical results, a systematic and quantitative comparison of such models remains to be explored. We propose in this contribution such a benchmark of road network growth models. We include in the comparison (i) a random null model; (ii) a random potential breakdown model (Raimbault, 2020); (iii) a deterministic potential breakdown model (Raimbault, 2019); (iv) a cost-benefit compromise model (Louf et al., 2013); (v) a biological network generation model (Raimbault, 2018); and (vi) a self- reinforcement model (Molinero and Hernando, 2020).

At each time step, with a fixed population density: 

\begin{enumerate}
	\item Add new nodes preferentially to population and connect them
	\item \justify Variable heuristic for new links, among: nothing, random, gravity-based deterministic breakdown, gravity-based random breakdown (from \cite{schmitt2014modelisation}), cost-benefits (from \cite{louf2013emergence}), biological network generation (based on \cite{tero2010rules})
\end{enumerate}

\medskip

\centering

\frame{\includegraphics[width=0.32\textwidth]{figures/example_nwgrowth_tick0.png}}
\frame{\includegraphics[width=0.32\textwidth]{figures/example_nwgrowth_tick2.png}}
\frame{\includegraphics[width=0.32\textwidth]{figures/example_nwgrowth_tick10.png}}

}


\sframe{Biological network generation}{

Model introduced by~\cite{tero2010rules}: exploration and reinforcement by a slime mould searching for ressources

\bigskip

\includegraphics[width=0.32\textwidth]{figures/slimemould_tick1}
\includegraphics[width=0.32\textwidth]{figures/slimemould_tick10}
\includegraphics[width=0.32\textwidth]{figures/slimemould_tick20}\\
\includegraphics[width=0.32\textwidth]{figures/slimemould_tick50}
\includegraphics[width=0.32\textwidth]{figures/slimemould_tick101}
\includegraphics[width=0.32\textwidth]{figures/slimemould_reseauFinal}\\

\medskip

\footnotesize
\textit{Application to the design of optimal bus routes in \cite{raimbault2018systemes}}

}





\sframe{Biological Network generation}{

Adding new links with biological heuristic:

\begin{enumerate}
	\item Create network of potential new links, with existing network and randomly sampled diagonal lattice
	\item Iterate for $k$ increasing ($k\in \{ 1,2,4 \}$ in practice) :
	\begin{itemize}
		\item Using population distribution, iterate $k\cdot n_b$ times the slime mould model to compute new link capacities
		\item Delete links with capacity under $\theta_d$
		\item Keep the largest connected component
	\end{itemize}
	\item Planarize and simplify final network
\end{enumerate}

\medskip

\centering

\frame{\includegraphics[width=0.6\textwidth]{figures/7-1-1-fig-networkgrowth-bioexample.jpg}}

\footnotesize

\textit{Intermediate stage for biological network generation}

}


\section{Implementation}


\sframe{Model parameters}{

\centering

\vspace{-0.2cm}
\begin{tabular}{|p{1.5cm}|c|c|c|c|c|}
  \hline
Heuristic & Param. & Name & Process & Domain & Default\\
  \hline
\multirow{5}{*}{Base}& $l_m$ & added links & growth & $[0;100]$ & $10$ \\\cline{2-6}
 & $d_G$ & gravity distance & potential & $]0;5000]$ & $500$ \\\cline{2-6}
 & $d_0$ & gravity shape & potential & $]0;10]$ & $2$ \\\cline{2-6}
 & $k_h$ & gravity weight & potential & $[0;1]$ & $0.5$ \\\cline{2-6}
 & $\gamma_G$ & gravity hierarchy & potential & $[0.1;4]$ & $1.5$ \\\hline
\multirow{2}{*}{Random}& $\gamma_R$ & random selection  & hierarchy & $[0.1;4]$ & $1.5$ \\\cline{2-6}
& $\theta_R$ & random threshold & breakdown & $[1;5]$ & $2$ \\\hline
Cost-benefits & $\lambda$ & compromise & compromise & $[0;0.1]$ & $0.05$ \\\hline
\multirow{2}{*}{Biological}& $n_b$ & iterations & convergence & $[40;100]$ & $50$ \\\cline{2-6}
& $\theta_b$ & biological th.& threshold & $[0.1;1.0]$ & $0.5$ \\\hline
\end{tabular}

}



\sframe{Model setup}{

% We use the GHSL dataset for functional urban areas worldwide and OpenStreetMap to extract real networks and population distributions for the 1000 largest urban areas,

\textbf{Synthetic setup: } rank-sized monocentric cities, simple connection with bord nodes to avoid bord effects 

\textbf{Real setup: } Population density raster at 500m resolution (European Union, from Eurostat)

\bigskip

\centering
\frame{\includegraphics[width=0.35\textwidth]{figures/coevol_example_synthsetup}}\hspace{0.1cm}
\frame{\includegraphics[width=0.35\textwidth]{figures/coevol_example-realsetup}}

\textbf{Stopping conditions: } fixed final time; fixed total population; fixed network size.

}


\sframe{Network Indicators}{

% and to compute corresponding values of diverse network measures (including betweenness and closeness centralities, accessibility, performance, diameter, density, average link length, average clustering coefficient).

Network Topology measured by:

\begin{itemize}
	\item Average betweenness and closeness centralities
	\item Efficiency (network pace relative to euclidian distance)
	\item Mean path length, diameter
\end{itemize}

}

\sframe{}{

% The models are integrated into the spatialdata scala library (Raimbault et al., 2020) and into the OpenMOLE software for model exploration and validation (Reuillon et al., 2013).

}






\section{Results}

\sframe{Example of generated networks}{

\centering

\includegraphics[width=0.9\textwidth]{figures/7-1-2-fig-networkgrowth-examples.jpg}

\footnotesize\textit{In order: connection; random; deterministic breakdown; random breakdown; cost-driven; biological.}

}


\sframe{Pattern Space Exploration algorithm}{

% We then run a diversity search algorithm, the Pattern Space Exploration algorithm (Cherel et al., 2015), for each model with their own free parameters and with the population distribution also as input parameter among the sampled areas. This algorithm is specifically tailored to provide feasible spaces of model outputs in relatively low dimensions.

\includegraphics[width=\linewidth]{figures/pone_0138212_g002.png}



}


\sframe{Indicators feasible space}{

% We thus proceed to a principal component analysis on real data points and project simulated values on the two first components, taken as objectives of the diversity algorithm. We obtain different shapes of feasible point clouds and corresponding effective degrees of freedom, some regions in the objective space reachable by a single model only, and a small number of urban areas which can not be approximated by the models.

\centering

\includegraphics[width=\linewidth]{figures/scatter_pse_allmodels.png}


}


\sframe{Hypervolumes}{

\begin{center}
\includegraphics[width=0.7\linewidth]{figures/pointclouds-overlap.png}
\end{center}



}

\section{Discussion}

\sframe{Discussion}{

% This quantitatively confirms the complementarity of diverse processes driving road network growth, and the need for a plurality of models to explain it.

}




\sframe{Conclusion}{


\justify

$\rightarrow$ 

\medskip

$\rightarrow$ 

\bigskip
\bigskip



\textbf{Open repositories}

\medskip

\url{https://github.com/JusteRaimbault/NetworkGrowth} 




}






%%%%%%%%%%%%%%%%%%%%%
\begin{frame}[allowframebreaks]
\frametitle{References}
\bibliographystyle{apalike}
\bibliography{biblio}
\end{frame}
%%%%%%%%%%%%%%%%%%%%%%%%%%%%










\end{document}






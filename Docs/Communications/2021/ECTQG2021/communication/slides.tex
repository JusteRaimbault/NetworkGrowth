

\input{header_slides.tex}

\begin{document}


\title[Benchmarking road network growth models]{Benchmarking road network growth models}

\author[Raimbault]{J.~Raimbault$^{1,2,3,\ast}$\\\medskip
$^{\ast}$\texttt{j.raimbault@ucl.ac.uk}
}

\institute[UCL]{$^{1}$Center for Advanced Spatial Analysis, University College London\\
$^{2}$UPS CNRS 3611 Complex Systems Institute Paris\\
$^{3}$UMR CNRS 8504 G{\'e}ographie-cit{\'e}s
}


\date[04/11/2021]{ECTQG 2021\\
Special Session: Exploration and validation of spatial simulation models\\
November 4th 2021
}

\frame{\maketitle}



\section{Introduction}

\sframe{Modeling road network growth}{


% Processes underlying the growth of road networks are diverse and complementary, as for example with the combination of self-organisation and top-down planning (Barthelemy et al., 2013). Multiple generative models, more or less parsimonious and data-driven, have been introduced in the literature to reproduce existing networks and provide potential explanations on main processes driving their growth

\textit{Road networks functionally shaping territories \cite{dupuy1987vers}}

\smallskip

$\rightarrow$ path-dependence and implications for sustainable territorial systems on multiple time scales

\bigskip

\textit{Multiple driving processes \cite{barthelemy2013self}}

\smallskip

$\rightarrow$ which models to understand such growth?

\bigskip

Examples of transportation network growth models:

\smallskip

\begin{itemize}
	\item Transportation governance \cite{raimbault2021introducing}	
	\item Investments in public transport \cite{cats2020modelling}
	\item Multi-modal networks \cite{cats2021multi}
	\item Bicycle networks design \cite{szell2021growing}
	\item Network morphogenesis \cite{tirico2018morphogenesis}
	\item Geometrical processes \cite{courtat2011mathematics}
\end{itemize}










}


\sframe{Research objective}{

% Whereas each model includes plausible mechanisms and often yields reasonable empirical results, a systematic and quantitative comparison of such models remains to be explored. We propose in this contribution such a benchmark of road network growth models.

$\rightarrow$ Models proposed in the literature are validated/explored in their own context, often not compared beyond null models

\bigskip

$\rightarrow$ Need for model benchmarks to build integrated theories


\bigskip
\bigskip

\textbf{Research objective:}

\medskip

\textit{Compare road network growth models with diverse types of processes, integrated into a common multi-modeling framework, with a focus on feasible space using a diversity search algorithm}



}


\section{Model}


\sframe{Road network generation multi-model}{


% Whereas each model includes plausible mechanisms and often yields reasonable empirical results, a systematic and quantitative comparison of such models remains to be explored. We propose in this contribution such a benchmark of road network growth models. We include in the comparison (i) a random null model; (ii) a random potential breakdown model (Raimbault, 2020); (iii) a deterministic potential breakdown model (Raimbault, 2019); (iv) a cost-benefit compromise model (Louf et al., 2013); (v) a biological network generation model (Raimbault, 2018); and (vi) a self- reinforcement model (Molinero and Hernando, 2020).

At each time step, with a fixed population density: 

\begin{enumerate}
	\item Add new nodes preferentially to population and connect them
	\item \justify Variable heuristic for new links, among: nothing, random, \textbf{gravity-based deterministic breakdown} \cite{raimbault2019second}, gravity-based random breakdown (\cite{schmitt2014modelisation}), cost-benefits (\cite{louf2013emergence}), \textbf{biological network generation} (\cite{tero2010rules})
\end{enumerate}

\medskip

\centering

\frame{\includegraphics[width=0.32\textwidth]{figures/example_nwgrowth_tick0.png}}
\frame{\includegraphics[width=0.32\textwidth]{figures/example_nwgrowth_tick2.png}}
\frame{\includegraphics[width=0.32\textwidth]{figures/example_nwgrowth_tick10.png}}

}


\sframe{Deterministic breakdown link addition}{

Model explored in \cite{raimbault2019second}

\bigskip

\begin{enumerate}
\item Gravity potential given by
\[
V_{ij}\left(d\right) \textrm{=} \left[ \left(1 - k_h\right) \textrm{+} k_h \cdot \left( \frac{P_i P_j}{P^2} \right)^{\gamma} \right]\cdot \exp{\left( -\frac{d}{r_g \left(1 \textrm{+} d\textrm{/}d_0\right)} \right)}
\]

\item $k\cdot N_L$ links are selected with lowest $V_{ij}\left(d_N\right)\textrm{/}V_{ij}\left(d_{ij}\right)$ (strong demand compared to offer), among which the $N_L$ links with this highest rate (lest costly links) are realised
\item Network is planarised
\end{enumerate}
}




\sframe{Biological network link addition}{

Model introduced by~\cite{tero2010rules}: exploration and reinforcement by a slime mould searching for ressources

\bigskip

\includegraphics[width=0.32\textwidth]{figures/slimemould_tick1}
\includegraphics[width=0.32\textwidth]{figures/slimemould_tick10}
\includegraphics[width=0.32\textwidth]{figures/slimemould_tick20}\\
\includegraphics[width=0.32\textwidth]{figures/slimemould_tick50}
\includegraphics[width=0.32\textwidth]{figures/slimemould_tick101}
\includegraphics[width=0.32\textwidth]{figures/slimemould_reseauFinal}\\

\medskip

\footnotesize
\textit{Application to the design of optimal bus routes in \cite{raimbault2018systemes}}

}





\sframe{Biological network link addition}{

Adding new links with biological heuristic:

\begin{enumerate}
	\item Create network of potential new links, with existing network and randomly sampled diagonal lattice
	\item Iterate for $k$ increasing ($k\in \{ 1,2,4 \}$ in practice) :
	\begin{itemize}
		\item Using population distribution, iterate $k\cdot n_b$ times the slime mould model to compute new link capacities
		\item Delete links with capacity under $\theta_d$
		\item Keep the largest connected component
	\end{itemize}
	\item Planarize and simplify final network
\end{enumerate}

\medskip

\centering

\frame{\includegraphics[width=0.55\textwidth]{figures/7-1-1-fig-networkgrowth-bioexample.jpg}}

\footnotesize

\textit{Intermediate stage for biological network generation}

}


\section{Implementation}


\sframe{Model parameters}{

\centering

\resizebox{0.95\linewidth}{!}{
\begin{tabular}{|p{1.5cm}|c|c|c|c|c|}
  \hline
Heuristic & Param. & Name & Process & Domain & Default\\
  \hline
\multirow{5}{*}{Base}& $l_m$ & added links & growth & $[0;100]$ & $10$ \\\cline{2-6}
 & $d_G$ & gravity distance & potential & $]0;5000]$ & $500$ \\\cline{2-6}
 & $d_0$ & gravity shape & potential & $]0;10]$ & $2$ \\\cline{2-6}
 & $k_h$ & gravity weight & potential & $[0;1]$ & $0.5$ \\\cline{2-6}
 & $\gamma_G$ & gravity hierarchy & potential & $[0.1;4]$ & $1.5$ \\\hline
\multirow{2}{*}{Random}& $\gamma_R$ & random selection  & hierarchy & $[0.1;4]$ & $1.5$ \\\cline{2-6}
& $\theta_R$ & random threshold & breakdown & $[1;5]$ & $2$ \\\hline
Cost-benefits & $\lambda$ & compromise & compromise & $[0;0.1]$ & $0.05$ \\\hline
\multirow{2}{*}{Biological}& $n_b$ & iterations & convergence & $[40;100]$ & $50$ \\\cline{2-6}
& $\theta_b$ & biological th.& threshold & $[0.1;1.0]$ & $0.5$ \\\hline
\end{tabular}
}


}



\sframe{Model setup}{

% We use the GHSL dataset for functional urban areas worldwide and OpenStreetMap to extract real networks and population distributions for the 1000 largest urban areas,

\textbf{Synthetic setup: } rank-sized monocentric cities, simple connection with bord nodes to avoid bord effects 

\textbf{Real setup: } Population density raster at 500m resolution (European Union, from Eurostat)

\textbf{Initial network:} skeleton connecting centres

\bigskip

\begin{center}
\frame{\includegraphics[width=0.3\textwidth]{figures/coevol_example_synthsetup}}\hspace{0.3cm}
\frame{\includegraphics[width=0.3\textwidth]{figures/coevol_example-realsetup}}
\end{center}

\textbf{Stopping conditions:} fixed network size or maximal steps

}


\sframe{Network Indicators}{

% and to compute corresponding values of diverse network measures (including betweenness and closeness centralities, accessibility, performance, diameter, density, average link length, average clustering coefficient).

Network Topology measured by:

\begin{itemize}
	\item Average betweenness and closeness centralities
	\item Efficiency (network pace relative to euclidian distance)
	\item Mean path length, diameter
\end{itemize}

}








\section{Results}

\sframe{Example of generated networks}{

\centering

\includegraphics[width=0.9\textwidth]{figures/7-1-2-fig-networkgrowth-examples.jpg}

\footnotesize\textit{In order: connection; random; deterministic breakdown; random breakdown; cost-driven; biological.}

}


\sframe{Pattern Space Exploration algorithm}{

% We then run a diversity search algorithm, the Pattern Space Exploration algorithm (Cherel et al., 2015), for each model with their own free parameters and with the population distribution also as input parameter among the sampled areas. This algorithm is specifically tailored to provide feasible spaces of model outputs in relatively low dimensions.

\begin{center}
\includegraphics[width=\linewidth]{figures/pone_0138212_g002.png}
\end{center}

\textit{Source: \cite{cherel2015beyond}}

}



\sframe{OpenMOLE integration}{

% The models are integrated into the spatialdata scala library (Raimbault et al., 2020) and into the OpenMOLE software for model exploration and validation (Reuillon et al., 2013).

$\rightarrow$ Models implemented in NetLogo (scala implementation in progress \cite{raimbault2020scala})

\medskip

$\rightarrow$ Integrated into OpenMOLE model exploration open source software \cite{reuillon2013openmole}

\medskip

\begin{center}
\includegraphics[height=0.13\textheight]{figures/iconOM.png}
\includegraphics[height=0.13\textheight]{figures/openmole.png}
\end{center}

\medskip

\textit{Enables seamlessly (i) model embedding; (ii) access to HPC resources; (iii) exploration and optimization algorithms (including PSE)}

\medskip

\url{https://openmole.org/}

\bigskip

$\rightarrow$ PSE run separately for each model for 5000 generations

}


\sframe{Indicators feasible space}{

% We thus proceed to a principal component analysis on real data points and project simulated values on the two first components, taken as objectives of the diversity algorithm. We obtain different shapes of feasible point clouds and corresponding effective degrees of freedom, some regions in the objective space reachable by a single model only, and a small number of urban areas which can not be approximated by the models.

\centering

\includegraphics[width=0.9\linewidth]{figures/scatter_pse_allmodels.png}


}


\sframe{Hypervolume intersections}{

\begin{center}
\includegraphics[width=0.7\linewidth]{figures/pointclouds-overlap.png}
\end{center}

\footnotesize

\textit{Directed relative overlaps between estimated hypervolumes of point clouds}


}

\section{Discussion}

\sframe{Discussion}{

% This quantitatively confirms the complementarity of diverse processes driving road network growth, and the need for a plurality of models to explain it.

$\rightarrow$ Complementarity of models to generate diverse networks: models not only with different purposes but also "output contexts"

\bigskip

$\rightarrow$ Comparison with real networks / calibration (work in progress on GHS FUAs, see \cite{raimbault2019urban} for OSM 50km windows)

\bigskip

$\rightarrow$ Extension with other models \cite{molinero2020model} \cite{tirico2018morphogenesis} (work in progress)

\bigskip

$\rightarrow$ Diversity search algorithms and dimensionality reduction (work in progress)

\bigskip

$\rightarrow$ Application: link between calibrated parameters and sustainability indicators? \cite{raimbault2018multi}

}




\sframe{Conclusion}{


\justify

$\rightarrow$ Complementarity of models and processes for road network growth (similar result for population density morphogenesis \cite{raimbault2020comparison}

\medskip

$\rightarrow$ Which models to integrate? Open science/platforms for model sharing and benchmarking; typology/classification of processes, models, disciplines?

\bigskip
\bigskip


\textbf{Use and contribute to OpenMOLE}

\url{https://openmole.org}


\medskip


\bigskip

\textbf{Open repository}


\url{https://github.com/JusteRaimbault/NetworkGrowth} 




}






%%%%%%%%%%%%%%%%%%%%%
\begin{frame}[allowframebreaks]
\frametitle{References}
\bibliographystyle{apalike}
\bibliography{biblio}
\end{frame}
%%%%%%%%%%%%%%%%%%%%%%%%%%%%










\end{document}





